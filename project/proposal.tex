\documentclass[11pt]{article}

\usepackage[utf8]{inputenc}
\usepackage[T1]{fontenc}

\usepackage{fullpage}

\usepackage{graphicx}
\usepackage{verbatim}
\usepackage{siunitx}

\usepackage[colorlinks=false,pdfborder={0 0 0}]{hyperref}
\usepackage[all]{hypcap}


\title{Astro 585 Project Codename: The Maxwell-Jüttner Distribution}
\author{Mock Simulations for HETDEX LAE Extraction to determine Completeness
and Contamination}



\begin{document}


\maketitle

\url{https://docs.google.com/document/d/1Fj2cz_puXKnwiutwCG_OQCaqD6-GgNUagenL2Re3VFc/edit?usp=sharing}


\section{Goal}

HETDEX will use on the order of $\num{800000}$ Lyman-$\alpha$
emitting galaxies (LAEs) at redshifts $\sim2<z<\sim3.5$ to measure the equation
of state of dark energy. Ly-$\alpha$ emits at \SI{1232}{\angstrom} in the rest
frame, which gets shifted to the range $\num{2500}-\SI{5000}{\angstrom}$ at
these redshifts.

In the local universe at redshifts $z<0.5$, the forbidden line doublett
[OII]3727 gets shifted into the same range. Indeed, there is a significant
population of OII-emitters in the local universe. This introduces contamination
of the sample of LAEs on the order of 20\% if no careful classification is
done.

It is therefore important to simulate the observations to test different
algorithms that classify galaxies according to LAEs at high redshift versus OII
emitters in the local universe.

This project proposal is about inserting fake LAEs and OII emitters into real
images, and extracting their photometry for others to test their classification
algorithms. A rudimentary code mostly written in python already exists.

Parallelization will come into play in two ways. First, this can be done
embarassingly parallel for $\sim\num{1000000}$ galaxies by distributing this
among several computing nodes.

However, another possibly interesting parallelization technique will be to
insert the fake LAEs and OII emitters into an image using GPUs. This will
involve drawing from a distribution of possible galaxy morphological shapes.
With a GPU the same morphology could then be placed into multiple images. Of
course, multiple galaxies can also be placed into one image, as long as there
aren't too many to step on each other.


\section{Inputs and Outputs}

The input is provided as a ASCII text file with RA, Dec, $z$ for each LAE and
another ASCII text file for the OII emitters. They have been created to
simulate the large-scale structure of the universe form a given power spectrum.

We will also need to draw from the luminosity function of LAEs and OII
emitters, which is provided by a ASCII file for each type of galaxy.

Lastly, there is a collection of images to use as a background testbed. These
are stored in FITS file format.

The output for each LAE or OII emitter will be its continuum magnitudes for
whatever continuum bands we have (K is being observed, g is planned) and
Ly-$\alpha$ line flux.

The final product after applying a classification algorithm should be RA, Dec,
continuum magnitudes, line flux, flux error, equivalent width (assuming it is
an LAE), probability of being an LAE, and probability of being an OII emitter.
The probability calculations are beyond the scope of this proposal, and is done
by collaborators at Rutgers. The result will then be used to calculate the
power spectrum, and compared with the original inputs (also by collaborators,
one of which is coming to Penn State at the end of March).


\section{Testing plan}

This is a testing plan. Parts of it will be used for the actual experiment. To
test the testing pipeline, we will start with the already existing pipeline as
a reference. However, some error estimations are not yet implemented. In
particular, HETDEX will give the position of a potential LAE to about
\SI{1}{arcsecond}. This error in the position will need to be taken into
account when measuring the photometry of the LAE candidate. Mastering the force
with YODA\footnote{Yet another Object Detection Application,
\url{http://www.as.utexas.edu/~drory/yoda/index.html}. YODA works similarly to
SExtractor, but is better with multi-band imaging.} has already been achieved.
YODA is made of C++ code. We will then use this as a testbed for a more complex
analysis that takes the error in the position into account. The precise
algorithm has yet to be developed.

A rudimentary, unoptimized pipeline already exists. It is largely written in
Python, but uses YODA for the object detection and photometry. First, we will
perform unit tests on drawing randomly from the luminosity functions. The error
can be estimated, so a unit test may sometimes fail with a predetermined
frequency. The we will create another unit test for inserting a fake galaxy
into an image. This will be an empty image, a slightly more populated image,
and a realtively crowded image. Finally, unit tests will be developed to see
how well YODA performs the extraction.

The serial implementation will function as a reference for the parallelization.


\section{Problem size}

As previously mentioned, HETDEX will detect about \num{800000} LAEs, and a few
million OII emitters. We are therefore planning on simulating on the order of
one million galaxies to test the extraction pipeline, and determine our
fraction of contaminants as well as incompleteness.

We will not need a large variety of background images, perhaps on the order of
a few tens. The pipeline is therefore not very memory intensive. However,
extraction and error analysis will required substantial computational
resources.


\section{Target Architecture, Programming Language, Libraries, Parallel
Programming Paradigm}

Several approaches are possible here. The simplest would be to just run the
pipeline concurrently on multiple processors. This would not require many
changes. However, there seems to be much more potential for running parts of
the pipeline on a GPU.

Much of the existing pipeline is written in python. As there is considerable
momentum for python in the collaboration, we most likely we will continue using
python to a large extent. However, core algorithms that require speed will be
done in plain C. The state python bindings to GPU is unclear to the author at
this stage. However, C libraries do exist that can be taken advantage of.


\end{document}
