\documentclass[11pt]{article}

\usepackage[utf8]{inputenc}
\usepackage[T1]{fontenc}

\usepackage{fullpage}

\usepackage{graphicx}
\usepackage{verbatim}
\usepackage{siunitx}

\usepackage[colorlinks=false,pdfborder={0 0 0}]{hyperref}
\usepackage[all]{hypcap}


\title{Astro 585: HW 3}
\author{Codename: The Maxwell-Jüttner Distribution}



\begin{document}


\maketitle

\section{Common Function Benchmarks}
My git repository is here: \url{https://github.com/hsgg/astro585}, clone URL
\url{https://github.com/hsgg/astro585.git}.

\section{File I/O}
First the program file, then the output:
\verbatiminput{2_io.jl}

And for creating a YAML file I used the file \path{2_makeyaml.py}:
\verbatiminput{2_makeyaml.py}

The output is:
\begin{verbatim}
<snip random vector fo 1024 elements>

# 2a
Print to stdout takes 0.270311566 seconds

# 2b,c
Writing: 0.101148976 seconds
Reading: 3.015733168 seconds
Reading: 0.001776553 seconds
Reading: 0.001718864 seconds

Writing to a file is about 10 times faster than to writing to
stdout. Initial reading is slow, but subsequent ones are amazingly fast,
probably because julia (and the OS) caches the read.

Writing2^20: 1.245572694 seconds
Reading2^20: 1.769823281 seconds
Reading2^20: 1.74360053 seconds
Reading2^20: 1.744750048 seconds

Yeah, cache ain't big enough no more.

# 2d
Writing2^20 in binary: 0.275027949 seconds
Reading2^20 in binary: 0.11283971 seconds
Reading2^20 in binary: 0.02341513 seconds
Reading2^20 in binary: 0.023484832 seconds

ASCII file size: 18.268983840942383 MB
Binary file size: 8.0 MB

Smaller, meaner, faster!

# 2e
Writing2^20 in hdf5: 0.816217353 seconds
Writing2^20 in hdf5: 0.234449809 seconds
Reading2^20 in hdf5: 0.074714466 seconds
Reading2^20 in hdf5: 0.012444531 seconds
Reading2^20 in hdf5: 0.011976035 seconds

HDF5 file size: 8.002532958984375 MB

Writing is slower than pure binary, but reading is faster, which
seems odd. Possibly the 'read()' call does some buffer management that the HDF5
library can avoid with more information.

# 2f
The ASCII format will have to read all elements, because each element
has a different length, and we need to find the delimiters. The pure binary
format will probably be fastest, since I would just calculate the position of
the relevant element (i * sizeof(Float64)) and then seek to it. HDF5 might be
as fast as the pure binary format, but I imagine that it would first read the
beginning of the file to find out the structure of the file, and then seek to
the relevant position.

# 2g
ASCII is human readable, and human editable. However, the simple
version used in this homework I would never use in any serious work, as it is
too easy to confuse what each number means. ASCII in general is not useful for
storing large amounts of data, as it takes more than twice as much space, and
is significantly slower. It may also loose some precision when converting to a
decimal representation. However, it may be useful for small data tables.

Using a pure binary file is simple, but not so great, since the precise format
is easily forgotten, and may change depending on which machine you run it.

HDF5 seems nice for large data sets, but as for the pure binary format you will
need a program to read and edit it, so I would not use it for configuration or
parameter files. It is likely precisely enough defined to be the same no matter
what machine you run your program on.

# 2h
The file 'knuth.yaml' was created with './2_makeyaml.py > knuth.yaml',
because yaml for julia cannot yet write a YAML file at this time.
Reading2^10 in YAML: 9.325846978 seconds
Reading2^10 in YAML: 0.334877953 seconds

At this time, YAML is not a serious contender for storing any amount
of data. Besides the first read time being, uhm, slow, the table is not really
human readable. However, it looks like a really nice configuration file
format.
\end{verbatim}


\section{Testing}
First the program file, then the output:
\verbatiminput{3_test.jl}

With output:

\begin{verbatim}
Get entropy...
# 3a
ASCII success! (with approximation)
Binary success! (exact)
ERROR: no method jldopen(Symbol)
while loading /home/hsgg/pennstate/585-comp/hw3/3_test.jl, in expression starting on line 58
\end{verbatim}


\section{Documentation}
I added some documentation to the function 'test\_rw()' in part 3 of the
homework.


\end{document}
